\section{Statement of Goals}
% A
\cvline{}{As the gig economy evolves, automating rote tasks is taking a higher priority in business value chains. Rigid robots have played a transformative role in this evolution. However, these robots have low transportable loads, large flexure torques, and the traditional stiffening process meant to eliminate these errors contribute  to a high load-to-weight ratio -- complicating the actuation system; this  hampers their use in sophisticated control strategies that require minimizing or eliminating load-dependent errors.  Enter soft robots -- exhibiting distributed deformation, bending and twisting capabilities, yet possessing hyper-redundancy in their configuration space. These make them capable of flexible manipulation in delicate workspaces; their minimal resistance to applied strain and limited load-to-weight ratio make them choice mechanisms in human-safety automation domains.
%\subsection{Research Mission}
\textbf{My research goal} is to harness the capacity of soft matter for robotics,  cleverly exploiting their intrinsic morphological computation properties in order to yield simplified control laws and human-friendly robot manipulation systems. I want to continue providing robust models (via analytic and AI methods) as well as controllers for soft robots that serve as better alternatives to current rigid manipulation technologies in \bf{medicine} and \bf{industrial automation}. Therefore, I am interested in designing
\begin{inparaenum}[(1)]
\item robust models,
\item robust adaptive controllers, and
\item software frameworks
\end{inparaenum}
that make it easier (for engineers, chemists, and biologists alike) to create, test, verify and validate soft continuum manipulators as designers originally envisioned.}{}

%\section{Previous and Current Research}
%\cvitem{2017-2018}{Building robustness into deep visuomotor policies with game theoretical approaches.}{}
%\cvitem{2014-Present}{Assuring patient comfort and mitigating dose attenuation during head and neck cancer treatment using soft robots.}{}
%\cvitem{2018-2019}{Beam Orientation Optimization in Cancer Radition Therapy Using Reinforcement Learning.}{}
%\cvitem{2019-Present}{Assuring patient comfort and mitigating dose attenuation during head and neck cancer treatment using a Stewart-Gough platform.}{}