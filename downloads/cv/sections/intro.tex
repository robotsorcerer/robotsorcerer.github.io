\section{Statement of Goals}
\subsection{Research Mission}
\cvline{}{As the gig economy develops, it is paramount to automate rote manufacturing tasks with affordable yet efficient technology. There is a unique advantage that soft robots offer for real-world manipulation that rigid robots lack: morphological computation. Properly leveraging this yields simplified control laws and mechanical design for manipulation tasks. The goal of my research is to provide adequately robust models and controllers for soft robots that serves as better alternatives to the pervasive robotic manipulation technologies  with large flexure torques and high load-to-weight ratios. I am interested in designing 
\begin{inparaenum}[(1)]
\item robust models,
\item robust adaptive controllers, and
\item software simulation frameworks tools
\end{inparaenum}
to make it easier for people to test, verify and create soft manipulable devices they intended.}{}

%\subsection{Outreach Mission}
%\cvline{}{Public understanding and consumption determine the impact of research. In addition to educating my students to understand the importance of programming languages research, I am interested in \emph{improving science communication} and \emph{facilitating commercialization of technical ideas}. From 2013-2015 I co-directed NeuWrite Boston, a working group of scientists and science writers. To narrow the gap between academia and industry, I co-founded the Cybersecurity Factory, an accelerator for security startups.}{}
%
%\subsection{Research Directions}
%\cvline{}{I am investigating how the following two programming models facilitate the creation of programs and analyses that were previously difficult or impossible.}{}
%\cvlistitem{\textbf{Policy-agnostic programming for security and privacy.} As an alternative to approaches for detecting information leaks, I propose a new programming model that factors out the specification of security and privacy concerns from the rest of the program and enforces the properties \emph{by construction}. In my prior work I designed a language semantics for policy-agnostic programming with \emph{informaton flow policies} and developed dynamic and static enforcement techniques. I am currently interested in
%\begin{inparaenum}[(1)]
%\item extending the approach for statistical privacy and
%\item techniques for retrofitting legacy code with the policy-agnostic model, for purposes of fixing bugs and interacting with new policies and code.
%\end{inparaenum}}
%\cvlistitem{\textbf{Rule-based programming for biological modeling.} Traditionally, researchers model intracellular signalling using systems of ordinary differential equations (ODEs), but there are two problems with ODE models. First, a precise model requires a different ODE for each interaction between agents, causing ODE models to scale poorly with respect to number of agent types. Second, ODE models have little structure that we can exploit for scale-mitigating analyses. As an alternative, rule-based languages allow the representation of models as programs describing rewrites over graphs, where nodes correspond to proteins and edges describe protein complexes. Not only are these programs more concise than the corresponding ODE systems, but their structure also supports various analyses that are otherwise not possible. My current work focuses on analyzing \emph{causal relationships} between rules, and combining causal information with language design and model-checking techniques to create biologically relevant model analyses.}
