\section{Statement of Goals}
% A
\cvline{}{Automating rote tasks has taken on higher priority in business value chains. Rigid robots have played a leading role in this economic transformation. However, rigid robots have low transportable loads, and large flexure torques which affect their positioning accuracy. To minimize positioning accuracy  e.g. error magnification from shoulder out to end-effector, their joints are  stiffened during the manufacturing process -- contributing  a high load-to-weight ratio; and hence complicating the actuation mechanism and hindering their use in sophisticated manipulation tasks.  In contrast, soft robots exhibit distributed deformation aided by hyper-redundancy and flexible manipulation capabilities in their configuration space. Their minimal resistance to applied strain and limited load-to-weight ratio make them choice mechanisms in human-robot automation domains given their compliance and hence their safety guarantees.
%\subsection{Research Mission}
\textbf{My research goal} is to harness soft matter for robots, exploiting their intrinsic morphological computation properties in order to yield simplified control laws and human-friendly robot manipulation systems. I want to continue providing robust models (via analytic and AI methods) as well as controllers for soft robots that serve as better alternatives to current rigid manipulation technologies in \bf{medicine} and \bf{industrial automation}.  I am interested in designing
\begin{inparaenum}[(1)]
\item robust models,
\item robust adaptive controllers, and
\item software frameworks
\end{inparaenum}
that make it easier (for engineers, chemists, and biologists alike) to create, verify and validate soft continuum manipulators as designers originally envisioned.}{}

%\section{Previous and Current Research}
%\cvitem{2017-2018}{Building robustness into deep visuomotor policies with game theoretical approaches.}{}
%\cvitem{2014-Present}{Assuring patient comfort and mitigating dose attenuation during head and neck cancer treatment using soft robots.}{}
%\cvitem{2018-2019}{Beam Orientation Optimization in Cancer Radition Therapy Using Reinforcement Learning.}{}
%\cvitem{2019-Present}{Assuring patient comfort and mitigating dose attenuation during head and neck cancer treatment using a Stewart-Gough platform.}{}